%% This is an example first chapter.  You should put chapter/appendix that you
%% write into a separate file, and add a line \include{yourfilename} to
%% main.tex, where `yourfilename.tex' is the name of the chapter/appendix file.
%% You can process specific files by typing their names in at the 
%% \files=
%% prompt when you run the file main.tex through LaTeX.
\chapter{Partition Function Computation and Improvements}
\section{Introduction}

As a quick review, the partition function for a thermodynamic system
of fixed volume, in contact with an environment with temperature $T$,
is

\begin{equation} Z = \sum_s e^{E(s)/ RT } \end{equation}

where $s$ denotes a particular state of the system, $E(s)$ is the
energy of that state, and $RT$ is the gas constant multiplied by the
temperature, specified above. Each particular term in the sum is
called that state's Boltzmann factor. The probability of a state is
then said to be its Boltzmann factor divided by the partition
function, or

\begin{equation} P(s) = \frac{e^{E(S)/RT}}{Z}.  \end{equation}

For an RNA molecule, we want to compute the probability of a
particular folding or group of foldings, so we treat it as a
thermodynamic system and sum up the energies of each state, which is a
particular folding. The energy that we assign to an RNA folding is
determined by the Turner Free energy model, mentioned in the
introduction. According to this model, the energy of an RNA folding is
the sum of the energies of the loops that are created by the
folding. These energies are added, for the most part, linearly. This
means if the partition function for some small segment of the strand
is computed, it will have the same contribution to the partition
function of any larger segment that contains it. So we can spare
ourselves from enumerating every single folding by using an approach
that saves the results of these sub-computations in a table, such as
dynamic programming.

The dynamic programming algorithm for computing the partition function
of an RNA strands has several versions. If you ignore psuedoknots, and
if you make an approximation that internal loops will never exceed a
certain length, there is a general agreement that the fastest
algorithm runs in $O(n^3)$, where $n$ is the length of the strand. We
believe that we can streamline this computation even more, taking
advantage of the fact that empirically, the number of probable base
pairs of a strand of length $n$ seems to grow like $n$, not
$n^2$. This is the same result we used to speed up the stochastic
traceback algorithm and [TODO: see if this actually works].

\section{Motivation}

In certain situations, such as partition function clustering, the
partition function is computed and recomputed several times. If the
partition function takes on the order of hours or days to compute,
this can make partition function clustering a bad option. However in
these situations it is also true that the partition function is
recomputed with almost the same properties, just certain pairs
restricted. This motivates a method of computing the partition
function using a known pairs heueristic to prune away unneccesary
computation.

This concept has already been implemented to great success in the
stochastic traceback algorithm. We've been able to show via experiment
that the partition function only admits roughly $O(n)$ pairs with
probabilities above thresholds around the machine precision limit. If
we have the partition function already computed, we can recompute it
by only adding in pairs that have sufficient probability. We can also
extend this method: if a good heuristic appears in the future, one
that can eliminate a large number of pairs, while being
computationally cheap, we should be able to use the results to speed
up the partition function computation.

\section{Computation}

The standard way of computing the partition function involves filling
out a table where the $(i,j)$ member represents the partition function
for the substrand from base $i$ to base $j$. Because the energy model
for RNA is (mostly) linear, the partition function from $i$ to $j$ can
be expressed as a function of nearby members of this table. This
function is the recurrence relation for the partition function of
RNA. Because the free energy model is so complicated and has gone
through many iterations, differenct RNA folding software packages
implement different versions of the recurrence relation, and they vary
widely in complexity.

The definitive representation of the recurrence relation for RNA was
formulated in 1990 by J.S. McCaskill in his landmark paper \emph{The
Equilibrium Partition Function and Base Pair Binding Probabilities for
RNA Secondary Structure} [TODO: cite?]. The formula is also presented
better and explained well by a later paper by Dirks and Pierce in 2003
(Dirks & Peirce 2003). Starting at the outermost layer of this
relation, the formula for the partition function of the strand from
base $i$ to base $j$ is:

\begin{equation} Q(i,j) = 1 + \sum_{i \leq d < e \leq j}Q(i, d - 1)
Q^b(d, e) \end{equation}

The theory behind this formula is that the partition function is a sum
of the empty state (the first term, 1) and the state with at least 1
pair, the furthest pair to the right being pair $(d,e)$. The term
$Q^b(d,e)$ is the partition function assuming that base $d$ and base
$e$ are paired. This function has the following recursion relation:

\begin{equation} Q^b(i, j) = e^{-\frac{\text{Hairpin}(i,j)}{RT}} +
\sum_{i \leq d < e \leq j} e^{\frac{\text{Interior}(i, d, e,
j)}{RT}}Q^b(d,e) + \sum_{i \leq d < e \leq j} Q^m(i + 1, d - 1)Q^b(d,
e) e^{-\frac{\alpha_1 + 2\alpha_2 + \alpha_3(j-e-1)}{RT}}
\end{equation}

The theory behind this formula is that the partition function for a
strand assuming $i$ and $j$ are paired includes 3 cases:

\begin{enumerate}

\item

There are no bases paired between $i$ and $j$, the loop is a hairpin
and uses the energy function for a hairpin loop, we call
$\text{Hairpin}(i,j)$, which consists of data table lookups.

\item There is an internal loop between $i$ and $j$ and a second pair
$d$ and $e$. This uses a different energy model, we call
$\text{Internal}(i,j)$ and also consists of data table lookups.

\item There is a multiloop formed by the pair $i$ and $j$, which must
be carefully accounted for using a special model for multiloops.

\end{enumerate}

The multiloop partition function, $Q^m(i, j)$ is the last piece of the
puzzle. The formula is:

\begin{equation} Q^m(i, j) = \sum_{i \leq d < e \leq j}
e^{-\frac{\alpha_2 + \alpha_3(d-i) + \alpha_3(j-e)}{RT}} Q^b(d,e) +
Q^m(i, d - 1)Q^b(d, e) e^{-\frac{\alpha_2 + \alpha_3(j-e)}{RT}}
\end{equation}

In english, this just means we sum up all the ways to just have 1
pair, and then all the ways to have more than one pair. The case with
no pairs is not included, as in the original recursion in $Q^b$,
$Q^m(i+1, d-1)Q^b(d,e)$ must yield at least 2 pairs. Since $Q^b$ makes
one, then $Q^m$ must make at least 1.

For example, the UNAFold software package implements a particularly
hairy recurrence relation. Define $Q(i,j)$ as the partition function
from $i$ to $j$, $Q'(i,j)$ to be the partition function from $i$ to
$j$, assuming $i$ and $j$ are paired, and define $Q^1(i,j)$ to be the
partition function from $i$ to $j$, assuming exactly 1 pair happens on
that interval, and that pair happens with base $i$. The recurrence
relation is therefore

[recurrence relation]

Note the terms $Z_{ND}$, $Z_{3'D}$, $Z_{5'D}$, and $Z_{DD}$ are extra
free energy terms corresponding to 'dangle energies' which are the
results of an experiment later implemented in the model to improve it
from the standard energy model. In addition there are AU penalty terms
appended to where pairs are made, as AU and GU pairs have penalties
associated with forming. These additional energy terms improve the
model's predictive ability and bring the model closer to the "truth",
however it unfortunately makes the partition function seem very
threatening.

Our new partition function relation has the following theory behind
it: Assume we have the functions $I : B \to \{B\}$ and $J : B \to
\{B\}$ that return the set of all probably pairs for a base $i$ or a
base $j$, respectively. The recurrence relation can be reformulated in
the following way:

\section{Derivation of new Q(i, j) formula}

In UNAfold, we have that the old recurrence relations were as follows:

\begin{equation}
Q(i,j) = \sum_{k=i}^j \left ( Q(i, k-1) + e^{-\frac{b(k-i)}{RT}}  \right )Q^1(k, j)
\end{equation}
where 
\begin{equation}
\begin{split}
Q^1(i, j) = \ \ & Q^1(i, j - 1) e^{-\frac{b}{RT}}  \\
 +\ & e^{-\frac{c}{RT} }Z_{ND}(i, j) Q'(i, j)  \\
+\ & e^{-\frac{b + c}{RT}}Z_{5'D}(i + 1, j)Q'(i + 1, j)  \\
+\ &  e^{-\frac{b + c}{RT}}Z_{3'D}(i, j-1)Q'(i, j - 1)  \\
+\ &  e^{-\frac{2b + c}{RT}}Z_{DD}(i + 1, j-1)Q'(i + 1, j-1) 
\end{split}
\end{equation}
\noindent
We can expand the recursive definition of $Q^1(i,j)$:

\begin{equation}
\begin{split}
Q^1(i, j) = \sum_{k' = i + 1}^j e^{-\frac{b(j - k')}{RT} } \bigg [ \ 
  & e^{-\frac{c}{RT} }Z_{ND}(i, k') Q'(i, k') \\
 +\ & e^{-\frac{b + c}{RT}}Z_{5'D}(i + 1, k')Q'(i + 1, k') \\ 
+\  & e^{-\frac{b + c}{RT}}Z_{3'D}(i, k'-1)Q'(i, k' - 1) \\
+\  & e^{-\frac{2b + c}{RT}}Z_{DD}(i + 1, k'-1)Q'(i + 1, k'-1) \   \bigg ]
\end{split}
\end{equation}
\noindent
Plugging this into $Q(i,j)$ we get:

\begin{equation}
\begin{split}
Q(i, j) = \sum_{k= i}^j\ \sum_{k' = k + 1}^j \left (  Q(i, k-1) + e^{-\frac{b(k-i)}{RT}} \right ) e^{-\frac{b(j - k')}{RT} } \bigg [ \ 
  & e^{-\frac{c}{RT} }Z_{ND}(k, k') Q'(k, k')  \\
+\  & e^{-\frac{b + c}{RT}}Z_{5'D}(k + 1, k')Q'(k + 1, k') \\ 
+\ & e^{-\frac{b + c}{RT}}Z_{3'D}(k, k'-1)Q'(k, k' - 1) \\
+\  & e^{-\frac{2b + c}{RT}}Z_{DD}(k + 1, k'-1)Q'(k + 1, k'-1) \   \bigg ]
\end{split}
\end{equation}
\noindent
Now we'll take the $j$th element of the second sum and split it out (note that the $j$th part of the 1st sum has no elements to sum now, so we can decrement that too):
\begin{equation}
\begin{split}
Q(i, j) = \sum_{k= i}^{j-1}\ \sum_{k' = k + 1}^{j-1} \left (  Q(i, k-1) + e^{-\frac{b(k-i)}{RT}} \right ) e^{-\frac{b(j - k')}{RT} } \bigg [ \ 
  & e^{-\frac{c}{RT} }Z_{ND}(k, k') Q'(k, k')  \\
+ \  & e^{-\frac{b + c}{RT}}Z_{5'D}(k + 1, k')Q'(k + 1, k')\\ 
+\   & e^{-\frac{b + c}{RT}}Z_{3'D}(k, k'-1)Q'(k, k' - 1) \\
+\  & e^{-\frac{2b + c}{RT}}Z_{DD}(k + 1, k'-1)Q'(k + 1, k'-1) \   \bigg ] \\
+\ \sum_{k=i}^j  \left (  Q(i, k-1) + e^{-\frac{b(k-i)}{RT}} \right ) \bigg [ \ 
  & e^{-\frac{c}{RT} }Z_{ND}(k, j) Q'(k, j) \\
+\  & e^{-\frac{b + c}{RT}}Z_{5'D}(k + 1, j)Q'(k + 1, j) \\ 
+\  & e^{-\frac{b + c}{RT}}Z_{3'D}(k, j-1)Q'(k, j - 1) \\
 +\  & e^{-\frac{2b + c}{RT}}Z_{DD}(k + 1, j-1)Q'(k + 1, j-1) \   \bigg ]
\end{split}
\end{equation}
\noindent
Notice that the double sum is simply $Q(i,j-1)e^{-b/RT}$ and the terms of the second, single sum are over the pairs with $j$ or $j-1$. Therefore, we can use our heuristic for the pairs of $j$ and $j-1$ to produce the following computation for $Q(i, j)$ which is much more efficient than the previous ones:
\begin{equation}
\begin{split}
Q(i,j) = Q(i, j-1)e^{-b/RT} +  \sum_{k(j)} & \bigg [  \left (  Q(i, k-1) + e^{-\frac{b(k-i)}{RT}} \right ) \
   e^{-\frac{c}{RT} }Z_{ND}(k, j) Q'(k, j)  \\
+\ & \left (  Q(i, k-2) + e^{-\frac{b(k-i-1)}{RT}} \right )    e^{-\frac{b + c}{RT}}Z_{5'D}(k, j)Q'(k, j) \   \bigg ]  \\
+\  \sum_{l(j-1)} & \bigg [  \left (  Q(i, l-1) + e^{-\frac{b(l-i)}{RT}} \right ) \
   e^{-\frac{c}{RT} }Z_{ND}(l, j-1) Q'(l, j-1)  \\
+\ & \left (  Q(i, l-2) + e^{-\frac{b(l-i-1)}{RT}} \right )   e^{-\frac{2b + c}{RT}}Z_{DD}(l, j-1)Q'(l, j-1) \   \bigg ] 
\end{split}
\end{equation}

\section{Derivation of new Q'(i, j) formula}
For $Q'(i, j)$ we start with the recursion:

\begin{equation}
\begin{split}
Q'(i,j) = Z_H(i, j) &+ Z_S(i, j) Q'(i+1, j-1) + QBI(i, j) \\
+\ & e^{-\frac{a+c}{RT}}Z_{ND}(j, i) \sum_{k = i + 3}^{j-5}Q(i+1, k - 1)Q^1(k, j-1)  \\
+\ & e^{-\frac{a+b+c}{RT}}Z_{3'D}(j, i) \sum_{k = i + 4}^{j-5}Q(i+2, k - 1)Q^1(k, j-1)  \\
+\ & e^{-\frac{a+b+c}{RT}}Z_{5'D}(j, i) \sum_{k = i + 3}^{j-6}Q(i+1, k - 1)Q^1(k, j-2) \\
+\ & e^{-\frac{a+2b+c}{RT}}Z_{DD}(j, i) \sum_{k = i + 4}^{j-6}Q(i+2, k - 1)Q^1(k, j-2) 
\end{split}
\end{equation}
\noindent
The 4 for loops in this make this an expensive computation as the number of bases gets very high. However, these for loops are very similar to the partition function in structure. Indeed, we could perhaps replace each of them with a function of the form $Q^m(i, j)$ defined as

\begin{equation}
Q^m(i, j) = \sum_{k = i +3}^{j-5} Q(i + 1, k - 1) Q^1(k, j - 1)
\end{equation} 
\noindent
Which would simplify the previous sum to a constant time computation, provided we have memoized $Q^m$:
\begin{equation}
\begin{split}
Q'(i,j) = Z_H(i, j) &+ Z_S(i, j) Q'(i+1, j-1) + QBI(i, j)  \\
+\ & e^{-\frac{a+c}{RT}}Z_{ND}(j, i) Q^m(i, j)  \\
+\ & e^{-\frac{a+b+c}{RT}}Z_{3'D}(j, i) Q^m( i + 1, j) \\
+\ & e^{-\frac{a+b+c}{RT}}Z_{5'D}(j, i) Q^m(i, j- 1) \\
+\ & e^{-\frac{a+2b+c}{RT}}Z_{DD}(j, i) Q^m(i + 1, j -1)
\end{split}
\end{equation}

\noindent
Now there just needs to be a way to efficiently compute $Q^m$. First we substitute in the expanded version of $Q^1$:

\begin{equation}
\begin{split}
Q^m(i, j) = \sum_{k = i + 3}^{j - 5}\  \sum_{k' = k + 1}^{j- 1} Q(i + 1, k - 1)  e^{-\frac{b(j - k')}{RT} } \bigg [ \ 
  & e^{-\frac{c}{RT} }Z_{ND}(k, k') Q'(k, k') \\
+\  & e^{-\frac{b + c}{RT}}Z_{5'D}(k + 1, k')Q'(k + 1, k') \\ 
 +\ & e^{-\frac{b + c}{RT}}Z_{3'D}(k, k'-1)Q'(k, k' - 1) \\
+\  & e^{-\frac{2b + c}{RT}}Z_{DD}(k + 1, k'-1)Q'(k + 1, k'-1) \   \bigg ]
\end{split}
\end{equation}

\noindent
Then we do as before and seperate out the $j$th term of the second sum. Note that there seems to be an additional sum needed to account that I've decreased the first sum's endpoint to $j-6$, but the sum ends up being from $k'= j-4$ to $k' = j -2$ and since $Q'$ for bases less than 4 apart is 0 due to hairpin loop rules, this sum is equal to zero.
\begin{equation}
\begin{split}
Q^m(i, j) = \sum_{k = i + 3}^{j - 6}\  \sum_{k' = k + 1}^{j- 2} Q(i + 1, k - 1)  e^{-\frac{b(j - k'-1)}{RT} } \bigg [ \ 
  & e^{-\frac{c}{RT} }Z_{ND}(k, k') Q'(k, k') \\
+ \  & e^{-\frac{b + c}{RT}}Z_{5'D}(k + 1, k')Q'(k + 1, k') \\ 
+ \  & e^{-\frac{b + c}{RT}}Z_{3'D}(k, k'-1)Q'(k, k' - 1) \\
 + \  & e^{-\frac{2b + c}{RT}}Z_{DD}(k + 1, k'-1)Q'(k + 1, k'-1) \   \bigg ]\\
+ \ \sum_{k = i + 3}^{j - 5}\ Q(i + 1, k - 1)  \bigg [ \ 
  & e^{-\frac{c}{RT} }Z_{ND}(k, j-1) Q'(k, j-1)  \\
+ \  & e^{-\frac{b + c}{RT}}Z_{5'D}(k + 1, j-1)Q'(k + 1, j-1) \\ 
  + \ & e^{-\frac{b + c}{RT}}Z_{3'D}(k, j-2)Q'(k, j-2)\\
+ \  & e^{-\frac{2b + c}{RT}}Z_{DD}(k + 1, j-2)Q'(k + 1, j-2) \   \bigg ]\ 
\end{split}
\end{equation}
\noindent
The double sum is again going to be equal to $Q^m(i, j -1)e^{-b/RT}$, and the second sum can be made much more efficient by our heuristic. 
\begin{equation}
\begin{split}
Q^m(i, j) = Q^m(i, j - 1)e^{-b/RT} + \sum_{k(j - 1)} \bigg [ & Q(i + 1, k - 1)   
  e^{-\frac{c}{RT} }Z_{ND}(k, j-1) Q'(k, j-1)  \\
 + \ &Q(i + 1, k - 2) e^{-\frac{b + c}{RT}}Z_{5'D}(k, j-1)Q'(k, j-1)  \bigg ]\ \\
+ \sum_{k(j-2)} \bigg [ & Q(i + 1, k - 1)   
  e^{-\frac{c}{RT} }Z_{3'D}(k, j-2) Q'(k, j-2)  \\
  +\ &Q(i + 1, k - 2) e^{-\frac{b + c}{RT}}Z_{DD}(k, j-2)Q'(k, j-2)  \bigg ]
\end{split}
\end{equation}
\noindent

Since the $k$s for any individual $j$ are found to be quite limited, the final form should be much more efficient at computing the $Q'(i, j)$.

Note that for $Q$ and in many places for $Q'$, instead of a sum over
the known $k$ that could possibly begin a leftmost pair, we see a
double sum. One of them over $k$ that could end a leftmost pair, and
this sum is limited to a certain length below $j$. This is just making
the same assumption that the internal loop computation makes: there
are not arbitrarily long strands without base pairs, after a certain
number of bases it becomes overwhelmingly more likely to make a base
pair that we can virtually ignore the energy of the the cases of
length beyond a certain $L$.

As for the seocnd sum, since the number of probable pairs for a base
$i$ has been shown empirically to be roughly constant, regardless of
length, the second sum is essentially constant. What this all means is
that all $O(n^2)$ computations of $Q(i,j)$'s are roughly constant
time. This means that the overall algorithm is $O(n^2)$, an
improvement over the previous algorithms asymptotic bound by and order
of $n$!



\section{results}

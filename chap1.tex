%% This is an example first chapter.  You should put chapter/appendix that you
%% write into a separate file, and add a line \include{yourfilename} to
%% main.tex, where `yourfilename.tex' is the name of the chapter/appendix file.
%% You can process specific files by typing their names in at the 
%% \files=
%% prompt when you run the file main.tex through LaTeX.
\chapter{Introduction}

The current free energy model of RNA is the result of evolution from
simple model. In the first iterations of the RNA, energy would be
determined by counting hydrogen bonds of cannonically paired. This
would mean that GC pairs are given three units, AU and GU pairs are
both given 2 [TODO: this is almost a quote from turner model paper,
not sure if I could word it differently, what to do?]. This was a
useful model to use as a baseline, and indeed it was used to design
the first minimum free energy algorithm [TODO: cite?]. However, it was
not very accurate, on average only 20.5% of known base pairs are
correctly predicted, and later energy models would use it as a control
for the hypothesis that they increased secondary structure prediction
accuracy (Mathews et al 1999).

Indeed, much improvement was made over the hydrogen bond model by
expanding it to include what are called sequence dependent
parameters. RNA is a polymer, something that bends and is flexible,
and that bending costs energy. Many of the nucleotides in a given RNA
sequence are not going to be paired to another base, and the energies
contributed by these bases as they make up the loops of the secondary
structure are going to be nonzero (unlike in the hydrogen bond
model). In fact, the energies of these loop regions are experimentally
found to be very different depending on what letters make up the
subsequence that defines the loop [Todo: find citation in Mathews
paper]. Thus, what the Turner model does is find the energies of all
the loop regions and add them up. Because the energy model treats the
loop energies as independent of one another it is called a (or the)
nearest neighbor model for RNA [TODO: this may not be precisely right,
consult prof aalberts].

[Todo: loop region figure]

The energy of a loop is dependent on what type of loop it
is. Different loops have different number of enclosing pairs, and this
has consequences in the energy model. For example, a base pair stack,
also called a helical region, is where we have two bases that are
adjacent to each other pairing to two other bases that are also
adjacent to each other. A general internal loop happens when there are
one or more unpaired bases in between the would-be adjacent base pairs
(no other pairs in between). An internal loop has a completely
different energy model compared to a helical region, even though they
are very similar. A multiloop can happen if we have a loop that is
enclosed by more than 2 base pairs, and this has a different energy
model still. In addition to these terms, energies associated with
unpaired bases next to paired bases, called dangling bases, are
included as well. Also a penalty for helices ending in AU's and
general miscellanous terms as more papers were written and more
energies were duct taped into the model [TODO: include kinder
wording]. The specifics of the energy calculations for different loops
are included in the following paragraphs, including how they are
derived from experiments.

\section{UV Melting Experiments}

The thermodynamic behavior of an RNA strand can be determined by
subjecting it to melting curve analysis. When a folded RNA strand
denatures, or unfolds because it is heated, its absorbance of UV
radiation changes. A physical model of this melting process can be
developed to interpret these curves.

\begin{equation}
\Delta G = \Delta H + T \Delta S
\end{equation}

\begin{equation}
T_M^{-1} = \frac{R}{\Delta H} \log{(C_T/a)} + \frac{\Delta S}{R}
\end{equation}

The free energy of a loop structure, or rather its $\Delta G$ relative
to the unfolded state,

\paragraph{Stacked Pairs}

The energy parameters for stacked pair loops were computed in a series
of optical melting experiments (estimating parameters by UV
absorbtion) by Xia et al (1998). For each combination of 2 sets of 2
paired bases, the change in energy at 37 Kelven was computed by
fitting $\Delta H$ and $\Delta S$ to the data. [Todo: decide whether
to have an in depth discussion of this]

For GU, a non-cannonical base pair that happens nontheless, the free
energy is calculated by subtracting the free energy of a CGUACG strand
from the free energy of a CGUUGACG strand, both of whose free energies
are determined by optical melting experiments.


\paragraph{Dangling ends and terminal mismatches}

[TODO: read serra & turner 1995]

bases adjacent to GU pairs are treated the same as bases adjacent to AU pairs

\paragraph{Hairpin loops}

The energy function for a hairpin loop is a similar table lookup.

[TODO: finish this]

- Tetraloop bonus

\paragraph{Bulge loops}

\paragraph{Internal loops}
- 2x2 tandem mismatches
- 2x1 internal loops
- 1x1 (single mismatch


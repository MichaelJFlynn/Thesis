%% This is an example first chapter.  You should put chapter/appendix that you
%% write into a separate file, and add a line \include{yourfilename} to
%% main.tex, where `yourfilename.tex' is the name of the chapter/appendix file.
%% You can process specific files by typing their names in at the 
%% \files=
%% prompt when you run the file main.tex through LaTeX.
\chapter{Conclusion}

My hopes for this thesis is that it will be of use to someone
interested in doing RNA structure research in the future. Predicting
RNA structure is a very interesting and difficult problem to solve,
but it also contains a variety of open problems that someone new to
the field can jump in and solve. I was very lucky to have been able to
jump in and take some of the low hanging fruit. In addition, I think
my research has opened up many more opportunities for future
researchers to grab more low hanging fruit.

One of these is combining Nestor and Barriers. Nestor is a great top
down approach that focuses on the highest energy conflict between
states to define the macrostates, and Barriers is a very good bottom
up approach that defines physically accurate and conceptually complete
thermodynamic variables. Combining the two approaches could be
fruitful.

Other improvements involve extending the efficiency improvements to
their natural generalization, which includes the pseudoknot
algorithm. This could be the most revolutionary change, because the
improvements to the standard partition function are less significant
since computers are fast enough to compute the largest strands in
reasonable time anyways. This is not true for pseudoknotted partition
functions, and an efficient algorithm could bring it into the realm of
the feasible. This is what is discussed in the next section. 

\section{Future work: Pseudoknot Algorithm}

The partition function algorithm discussed in this thesis and the
literature in general has a crucial flaw that makes it incomplete. The
algorithm assumes that RNA secondary structure is `well nested', that
there are no overlapping pairs $(i, j)$ and $(k, l)$ such that one of
$k$ or $l$ is between $i$ and $j$ and the other is not.

It is not true that RNA secondary structures are well-nested in
general. When a non-nested pair is present in a secondary structure,
that structure is said to contain a `pseudoknot'. Large groups of
classified RNA sequences such as Group I Introns (around 6\%
pseudeoknotted) and RNase P (around 13\% pseudoknotted) have
pseudoknots in their chemically determined secondary structure, and it
is in these groups that secondary structure prediction performs the
worst (Matthews et all 1999). Large RNA molecules tend to have
pseudoknots, which is unfortunate because the partition function
computation scales much worse when pseudoknots are included.

In general, computation of pseudoknotted structures is hard. In fact,
the problem has been shown to be NP-Complete. Algorithms
to perform the computation for a signficiant subset of pseudoknots
have been designed, most notably by Dirks and Pierce, who have
specified an $O(n^8)$ algorithm and a $O(n^5)$ simplification of that
algorithm, using the same approximations used to reduce the complexity
of the internal loop energy computation.

I believe the heuristic approach of only iterating over allowed pairs
will be able to significantly speed up the pseudoknot
computation. This could be truly revolutionary, because there is no
software that currrently includes pseudoknots in the computation of
the partition function. Since the algorithms seem to perform the worst
in high-pseudoknot structure groups, this could lead to a significant
improvement in RNA structure prediction. 

However, this would require a good energy model for pseudoknots to be
developed. There currently is only one pseudoknot energy model, it is
in the Dirks and Peirce paper \cite{dirks2003partition}. However this
energy model is not physically based, and is just a linear energy
model with a penalty for starting a pseudoknot and a constant bonus
for adding pairs. As we have seen with the standard energy model, this
approach does not capture enough interaction between terms, so a new
energy model needs to be specified. This could be another low hanging
fruit. 

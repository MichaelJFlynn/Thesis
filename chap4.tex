%% This is an example first chapter.  You should put chapter/appendix that you
%% write into a separate file, and add a line \include{yourfilename} to
%% main.tex, where `yourfilename.tex' is the name of the chapter/appendix file.
%% You can process specific files by typing their names in at the 
%% \files=
%% prompt when you run the file main.tex through LaTeX.
\chapter{Nestor and Clustering}
\section{Motivation}

Given that we know, for any given RNA strand, the probability of an
individual state is very low [TODO: reference section], a much more
important computation is the overall shape of the strand's free energy
landscape. Even if the probability of an individual state is low, if
we "integrate" over a basin of free energy, the probability of that
set of states could be something tangible.

In the past 10 years, several groups have started to explore this
concept. There are two approaches, in general, to define basins and
classify structures into them. The first class of methods defines the
basins from the top-down: given a number of stochastically sampled
structures, we divide them into groups based on some kind of distance
metric. These methods tend to be very similar to typical clustering
algorithms used in computer science and data analysis.

Another approach is to start at local minima and climb up the energy
barriers between minima using the metropolis-hastings algorithm to
maintain the correct probabilities according to the partition
function. These methods can be used to accurately compute the energies
of the transition states between local minima and these can tell you
the kinetics of the structure. This technique was developed by [TODO:
find Vienna people and cite them].

 Enabled by their stochastic sampling algorithm Ye Ding and
Charles E. Lawrence clustering algorithms. [TODO: evaluate this].
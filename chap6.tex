%% This is an example first chapter.  You should put chapter/appendix that you
%% write into a separate file, and add a line \include{yourfilename} to
%% main.tex, where `yourfilename.tex' is the name of the chapter/appendix file.
%% You can process specific files by typing their names in at the 
%% \files=
%% prompt when you run the file main.tex through LaTeX.
\chapter{Pseudoknot Algorithm}
\section{Introduction}

The partition function algorithm discussed in this thesis and the
literature in general has a crucial flaw that makes it incomplete. The
algorithm assumes that RNA secondary structure is `well nested', that
there are no overlapping pairs $(i, j)$ and $(k, l)$ such that one of
$k$ or $l$ is between $i$ and $j$ and the other is not.

[TODO: include figure of nestedness vs non-nestedness]

It is not true that RNA secondary structures are well-nested in
general. When a non-nested pair is present in a secondary structure,
that structure is said to contain a `pseudoknot'. Large groups of
classified RNA sequences such as Group I Introns (around 6\%
pseudeoknotted) and RNase P (around 13\% pseudoknotted) have
pseudoknots in their chemically determined secondary structure, and it
is in these groups that secondary structure prediction performs the
worst (Matthews et all 1999). Large RNA molecules tend to have
pseudoknots, which is unfortunate because the partition function
computation scales much worse when pseudoknots are included.

In general, computation of pseudoknotted structures is hard. In fact,
the problem has been shown to be NP-Complete [TODO: cite]. Algorithms
to perform the computation for a signficiant subset of pseudoknots
have been designed, most notably by Dirks and Pierce, who have
specified an $O(n^8)$ algorithm and a $O(n^5)$ simplification of that
algorithm, using the same approximations used to reduce the complexity
of the internal loop energy computation.

Our improvements bring a new concept to the computation of the
pseudoknot partition function: first do the unpseudoknotted partition
function computation, then redo the partition function including
pseudoknots, filtering the allowed pairs by their probabilities
according to the first computation. This approach basically combines a
heuristic and dynamic programming approach.

[TODO: compile pseudoknot recursion figures]

\section{Derivation}

The derivation of our pseudoknot recursion follows the path of our
original. The partition function is recursive in the same way, except
now there is a term for bases $i$ and $j$ containing a pseudoknot,
$Q^p(i,j)$, to go along with the standard paired region $Q^b(i,j)$. 

[TODO: straighten out internal loop penalities vs external loop penalties]
\begin{equation}
Q(i,j) = e^{- \beta b (j - i)} + \sum_{k,l} Q(i, k-1) \left ( Q^b(k, l) + Q^p(k, l)  \right )
\end{equation}

For a paired region we can now have a hairpin loop, an interrior loop,
a paired region with an undetermined region, one pseudoknot, or one
pseudoknot with an undetermined region.

\begin{equation}
\begin{split}
Q^b(i,j) =& Q^{HAIRPIN}(i, j) + QBI(i, j)  + QPI(i, j)\\
 +& \sum_{k,l} Q(i, k-1)( Q^b(k, l) + Q^p(k, l) ) \\ 
\end{split}
\end{equation}

For a pseudoknotted region, we must account for every way we could
pseudoknot a region, this includes summing over 6 indices (see
figure). These include new terms for the undetermined region inside a
pseudoknot $Q^z(i,j)$ and a pseudoknot internal loop $Q^b(i, k, l, j)$.

\begin{equation}
Q^p(i, j) = \sum_{a,b,c,d,e,d,f} Q^g(i, a, d, e) Q^g(b, c, f, j)
Q^z(a+1, b-1) Q^z(c+1, d-1) Q^z(e+1, f-1)
\end{equation}

The undertermined loop inside a pseudoknot, $Q^z(i,j)$, is the same as
$Q(i,j)$, except with energy penalties set by the pseudoknot model. 

\begin{equation}
Q^z(i,j) = e ^{-\beta b_p (j-i)} + \sum_{k,l} Q^z(i,j) Q^b(i,j)
\end{equation}

The last term is the internal loop for the pseudoknot (see figure):

\begin{equation}
Q^g(i, k, l, j) = Q^{INTERNAL}(i,j,k,l) +  \sum_{a,b} Q(i+1, a) Q^g(a, k, l, b) Q(l+1, j -1) 
\end{equation}



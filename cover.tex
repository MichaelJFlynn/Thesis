% -*-latex-*-
% 
% For questions, comments, concerns or complaints:
% thesis@mit.edu
% 
%
% $Log: cover.tex,v $
% Revision 1.8  2008/05/13 15:02:15  jdreed
% Degree month is June, not May.  Added note about prevdegrees.
% Arthur Smith's title updated
%
% Revision 1.7  2001/02/08 18:53:16  boojum
% changed some \newpages to \cleardoublepages
%
% Revision 1.6  1999/10/21 14:49:31  boojum
% changed comment referring to documentstyle
%
% Revision 1.5  1999/10/21 14:39:04  boojum
% *** empty log message ***
%
% Revision 1.4  1997/04/18  17:54:10  othomas
% added page numbers on abstract and cover, and made 1 abstract
% page the default rather than 2.  (anne hunter tells me this
% is the new institute standard.)
%
% Revision 1.4  1997/04/18  17:54:10  othomas
% added page numbers on abstract and cover, and made 1 abstract
% page the default rather than 2.  (anne hunter tells me this
% is the new institute standard.)
%
% Revision 1.3  93/05/17  17:06:29  starflt
% Added acknowledgements section (suggested by tompalka)
% 
% Revision 1.2  92/04/22  13:13:13  epeisach
% Fixes for 1991 course 6 requirements
% Phrase "and to grant others the right to do so" has been added to 
% permission clause
% Second copy of abstract is not counted as separate pages so numbering works
% out
% 
% Revision 1.1  92/04/22  13:08:20  epeisach

% NOTE:
% These templates make an effort to conform to the MIT Thesis specifications,
% however the specifications can change.  We recommend that you verify the
% layout of your title page with your thesis advisor and/or the MIT 
% Libraries before printing your final copy.
\title{RNA Macrostates and Computation Tools}

\author{Michael Flynn}
% If you wish to list your previous degrees on the cover page, use the 
% previous degrees command:
%       \prevdegrees{A.A., Harvard University (1985)}
% You can use the \\ command to list multiple previous degrees
%       \prevdegrees{B.S., University of California (1978) \\
%                    S.M., Massachusetts Institute of Technology (1981)}
\department{Department of Phyiscs}

% If the thesis is for two degrees simultaneously, list them both
% separated by \and like this:
% \degree{Doctor of Philosophy \and Master of Science}
\degree{Bachelors of Art in Physics}

% As of the 2007-08 academic year, valid degree months are September, 
% February, or June.  The default is June.
\degreemonth{June}
\degreeyear{2015}
\thesisdate{May 18, 2015}

%% By default, the thesis will be copyrighted to MIT.  If you need to copyright
%% the thesis to yourself, just specify the `vi' documentclass option.  If for
%% some reason you want to exactly specify the copyright notice text, you can
%% use the \copyrightnoticetext command.  
%\copyrightnoticetext{\copyright IBM, 1990.  Do not open till Xmas.}

% If there is more than one supervisor, use the \supervisor command
% once for each.
\supervisor{Daniel Aalberts}{Professor of Physics}

% This is the department committee chairman, not the thesis committee
% chairman.  You should replace this with your Department's Committee
% Chairman.
\chairman{Arthur C. Smith}{Chairman, Department Committee on Graduate Theses}

% Make the titlepage based on the above information.  If you need
% something special and can't use the standard form, you can specify
% the exact text of the titlepage yourself.  Put it in a titlepage
% environment and leave blank lines where you want vertical space.
% The spaces will be adjusted to fill the entire page.  The dotted
% lines for the signatures are made with the \signature command.
\maketitle 

% The abstractpage environment sets up everything on the page except
% the text itself.  The title and other header material are put at the
% top of the page, and the supervisors are listed at the bottom.  A
% new page is begun both before and after.  Of course, an abstract may
% be more than one page itself.  If you need more control over the
% format of the page, you can use the abstract environment, which puts
% the word "Abstract" at the beginning and single spaces its text.

%% You can either \input (*not* \include) your abstract file, or you can put
%% the text of the abstract directly between the \begin{abstractpage} and
%% \end{abstractpage} commands.

% First copy: start a new page, and save the page number.
\cleardoublepage
% Uncomment the next line if you do NOT want a page number on your
% abstract and acknowledgments pages.
% \pagestyle{empty}
\setcounter{savepage}{\thepage}
%\begin{abstractpage}
%% $Log: abstract.tex,v $
% Revision 1.1  93/05/14  14:56:25  starflt
% Initial revision
% 
% Revision 1.1  90/05/04  10:41:01  lwvanels
% Initial revision
% 
%
%% The text of your abstract and nothing else (other than comments) goes here.
%% It will be single-spaced and the rest of the text that is supposed to go on
%% the abstract page will be generated by the abstractpage environment.  This
%% file should be \input (not \include 'd) from cover.tex.
In this thesis, I designed and implemented a compiler which performs
optimizations that reduce the number of low-level floating point operations
necessary for a specific task; this involves the optimization of chains of
floating point operations as well as the implementation of a ``fixed'' point
data type that allows some floating point operations to simulated with integer
arithmetic.  The source language of the compiler is a subset of C, and the
destination language is assembly language for a micro-floating point CPU.  An
instruction-level simulator of the CPU was written to allow testing of the
code.  A series of test pieces of codes was compiled, both with and without
optimization, to determine how effective these optimizations were.

%\end{abstractpage}

% Additional copy: start a new page, and reset the page number.  This way,
% the second copy of the abstract is not counted as separate pages.
% Uncomment the next 6 lines if you need two copies of the abstract
% page.
% \setcounter{page}{\thesavepage}
% \begin{abstractpage}
% % $Log: abstract.tex,v $
% Revision 1.1  93/05/14  14:56:25  starflt
% Initial revision
% 
% Revision 1.1  90/05/04  10:41:01  lwvanels
% Initial revision
% 
%
%% The text of your abstract and nothing else (other than comments) goes here.
%% It will be single-spaced and the rest of the text that is supposed to go on
%% the abstract page will be generated by the abstractpage environment.  This
%% file should be \input (not \include 'd) from cover.tex.
In this thesis, I designed and implemented a compiler which performs
optimizations that reduce the number of low-level floating point operations
necessary for a specific task; this involves the optimization of chains of
floating point operations as well as the implementation of a ``fixed'' point
data type that allows some floating point operations to simulated with integer
arithmetic.  The source language of the compiler is a subset of C, and the
destination language is assembly language for a micro-floating point CPU.  An
instruction-level simulator of the CPU was written to allow testing of the
code.  A series of test pieces of codes was compiled, both with and without
optimization, to determine how effective these optimizations were.

% \end{abstractpage}



\section*{Acknowledgments}

I owe gratitude to a great many people. Thank you to my thesis advisor
Daniel Aalberts for accepting me as your thesis student and advising
me through the long process of writing a thesis. With your help and
encouragement I felt like I was able to make an impact in the field as
an undergraduate. I'd like to thank the faculty of the Physics
department of Williams College who have instructed me including Bill
Wootters, Kevin Jones, David Tucker-Smith, Ward Lopes, Michael
Seifert, Frederick Strauch, and Charlie Doret for their patience,
support and incredibly lucid explanations. I'd also like to thank the
Computer Science faculty, including Tom Murtagh, Jeannie Albrect,
Brent Heeringa, Morgan McGuire, Duane Bailey, Bill Lenhart, Stephen
Freund, and Brent Yorgey for their patience as well, and for making
some of my favorite classes at Williams.

I'd like to thank Michael Zuker and David Mathews for their
correspondance and help getting started in the field of RNA secondary
structure research. Thank you to Nick Markham for writing a very clear
Ph. D. thesis. 

I'd like to thank my friends including Maoli Vizcaino for keeping me
company while I toil, Tony Blanco, Qadir Forbes, JL Etienne and Dan
Evangelakos for their support, good coversations, and friendship. 

Most of all I'd like to thank my parents and my family for always
being there. No matter happens, I know I can come home to open arms.

\newpage

\section*{Executive Summary}

Standard algorithms in computational biology compute the minimum free
energy state of RNA. RNA is a polymer molecule that folds back on
itself because its components chemically bind to one another,
therefore the energy lanscape is very bumpy, with many local minima to
get stuck in and cause disagreement with these predictions. Predicting
the behavior of RNA based on macrostates, which are sets of states
clustered around local minima, is a better tool for RNA secondary
structure prediction. My research consisted of creating tools to
identify macrostates, and improving upon algorithms for tools that
already exist: the partition function algorithm and the stochastic
traceback algorithm.

Chapter 1 is an overview of the history of the field. It presents the
energy model of RNA, which underlies all computational prediction of
its thermodynamic variables. It is also large, complex, and important
to understand in order to understand the algorithms in later chapters.

Chapter 2 presents the algorithm to compute the partition function of
RNA and my improvements to that algorithm, with the assistance of an
allowed base-pairs heuristic. I show that one can signficantly speed
up the computation this way using computer timing tests. 

Chapter 3 presents the algorithm for sampling secondary structures
according to the Boltzmann ensemble, called the stochastic traceback
algorithm, and my improvments to that algorithm. These improvements
mirror those in chapter 2, with the same kind of timing tests.

Chapter 4 defines macrostates and examines two historical methods of
classifying and computing macrostates including Ding and Lawrence's
Sfold and Barriers from ViennaRNA. I then present Nestor, some past
work from Bill Jannen with my advisor Daniel Aalberts. Then I present
my own version of Nestor, splitting clusters based on partition
function results, which I call the PF method of Nestor. Finally I
describe SHAPE analysis on RNA, our interpretation and processing of
that data, and compare the results found with the PF method of Nestor
to the output of that processing. 

Chapter 5 concludes and presents some future work, including doing the
same optimizations mentioned in chapters 2 and 3 for the pseudoknot
algorithm, which could greatly expand the predictive ability of
secondary structure algorithm.

At the end of this thesis, I hope that the reader has a very good
understanding of the historical development of RNA secondary structure
prediction algorithm, a good grasp on the algorithms especially my
improvements, and a good idea of where further research could be done
to push the field forward.


%%%%%%%%%%%%%%%%%%%%%%%%%%%%%%%%%%%%%%%%%%%%%%%%%%%%%%%%%%%%%%%%%%%%%%
% -*-latex-*-
